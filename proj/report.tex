\documentclass[12pt]{article}
\usepackage[utf8]{inputenc}
\usepackage[T2A]{fontenc}
\usepackage{hyperref}
%\usepackage[warn]{mathtext}
\usepackage[english,russian]{babel}
\usepackage{graphicx}
\usepackage{amsmath}
\usepackage{amssymb}
\usepackage{amsthm}
\usepackage{caption}
\usepackage{tikz}
\usepackage{subcaption}
\usepackage{imakeidx}
\usepackage[russian]{cleveref}
\usepackage{enumitem}
\usepackage{mathtools}
\usepackage[normalem]{ulem} 
\usepackage{listings}
\usepackage{libertine}
\usepackage{polynom}
%\usepackage{MnSymbol}
%\usepackage{wasysym}
%\DeclareSymbolFont{T2Aletters}{T2A}{cmr}{m}{it}
\usepackage[a4paper,left=15mm,right=15mm,top=30mm,bottom=20mm]{geometry}
\parindent=0mm
\parskip=3mm
\usepackage{graphicx}
\renewcommand{\le}{\leqslant}

\renewcommand{\ge}{\geqslant}

\renewcommand{\phi}{\varphi}

\renewcommand{\epsilon}{\varepsilon}

\renewcommand{\qed}{\blacksquare}

\newcommand{\lar}{\Leftrightarrow}

\makeindex
\pagestyle{empty}

\newcommand{\Obig}{\mathcal{O}}

\newcommand{\verteq}{\rotatebox{90}{$\,=$}}
\newcommand{\equalto}[2]{\underset{\scriptstyle\overset{\mkern4mu\verteq}{#2}}{#1}}

\DeclarePairedDelimiter\ceil{\lceil}{\rceil}
\DeclarePairedDelimiter\floor{\lfloor}{\rfloor}



\def\binomtb#1#2{\ensuremath{\left(\kern-.3em\left(\genfrac{}{}{0pt}{}{#1}{#2}\right)\kern-.3em\right)}}
\newcommand{\ans}{\textbf{Ответ: }}

\newcommand{\then}{\Rightarrow}
\newcommand{\ifff}{$\iff$}


\newcommand{\lb}{\left}
\newcommand{\rb}{\right}

\definecolor{dkgreen}{rgb}{0,0.4,0}
\definecolor{gray}{rgb}{0.5,0.5,0.5}
\definecolor{lightgray}{rgb}{0.95,0.95,0.95}
\definecolor{mauve}{rgb}{0.58,0,0.82}

\def\commentstyle{\color{dkgreen}}
\lstset{ %
	language=C++,                   % the language of the code
	basicstyle=\footnotesize\ttfamily, % the size of the fonts that are used for the code
	numbers=left,                   % where to put the line-numbers
	numberstyle=\footnotesize\color{black},  % the style that is used for the line-numbers
	stepnumber=1,                   % the step between two line-numbers. If it's 1, each line 
	% will be numbered
	numbersep=0.7em,                % how far the line-numbers are from the code
	backgroundcolor=\color{lightgray}, % choose the background color. You must add \usepackage{color}
	showspaces=false,               % show spaces adding particular underscores
	showstringspaces=false,         % underline spaces within strings
	showtabs=false,                 % show tabs within strings adding particular underscores
	frame=single,                   % adds a frame around the code
	rulecolor=\color{black},        % if not set, the frame-color may be changed on line-breaks within not-black text (e.g. commens (green here))
	tabsize=2,                      % sets default tabsize to 2 spaces
	breaklines=true,                % sets automatic line breaking
	breakatwhitespace=false,        % sets if automatic breaks should only happen at whitespace
	identifierstyle=\color{blue!25!black},  
	keywordstyle=\color{blue!90!black},      % keyword style
	commentstyle=\commentstyle,     % comment style
	stringstyle=\color{mauve},      % string literal style
	escapeinside={\`}{\`},          % if you want to add a comment within your code
	escapebegin=\commentstyle\footnotesize,
	%morekeywords={n,k},             % if you want to add more keywords to the set
	morecomment=[l][\color{dkgreen}]{\#}, % to color #include<cstdio> 
	morecomment=[s][\commentstyle\color{gray!50!black}]{/**}{*/}
}

\usetikzlibrary{shapes,decorations,arrows,calc,arrows.meta,fit,positioning, automata}
\tikzset{
	-Latex,auto,node distance =1 cm and 1 cm,semithick,
	state/.style ={ellipse, draw, minimum width = 0.7 cm},
	point/.style = {circle, draw, inner sep=0.04cm,fill,node contents={}},
	bidirected/.style={Latex-Latex},
	el/.style = {inner sep=2pt, align=left, sloped}
}

\newcommand{\Let}{%
	\begin{tikzpicture}[scale=0.85]
	\draw[-] (0,0) -- (1ex,0);
	\draw[-] (1ex,0) -- (1ex, 2ex);
	\draw[-] (1ex,2ex) -- (0,2ex);
	\end{tikzpicture}%
	\;
	\hspace{0.001pt}
}


\newcommand{\dx}{\ dx}
\newcommand{\dt}{\ dt}
\newcommand{\dk}{\ dk}


\newcommand{\xor}{\oplus}

\makeindex
\pagestyle{empty}

\title{Проект по Формальным языкам. \\Многострочные комментарии.\\
 }
\author{Юра Худяков}

\begin{document}
	
	\maketitle

    Я решил поддержать возможность многострочных комментариях в парсерах пролога.

    Синтаксис многострочных комментариев: $(*\ text\ *)$

    При этом необходим контроль вложенности: $text$ может быть любым, и также содержать вложенные многострочные комментарии, при условии, что будет соблюдаться парность скобок. 

    Например, $(*\ (*\ hello\ world\ *)\ *)$ является корректным комментарием, а \[\begin{aligned} (*\ (*\ hello\ world\ *)\\ (*\ hello\ world\ *)\ *)\\ (*\ hello\ world\\ *)\\\end{aligned}\] не являются (на каждой строке отдельный комментарий)

    Несмотря на то, что в примерах выше рассматриваются однострочные варианты комментариев, многострочность поддерживается (на этот раз, это единый комментарий):

    \[
        \begin{aligned}
        (* \\
        hello\\
        (*\ (*\ ooo\ *)\ *) \\
        (*\\
        \\
        \\
        text\ *)\\
        world \\
        *)
    \end{aligned}
    \]

    \newpage

    Я реализовал поддержку этого синтаксиса в 3 парсерах

	\begin{enumerate}
		\item
            Парсер, основанный на рекурсивном спуске

            В этом парсере используется лексер Alex, поэтому поддержка реализована следующим образом:

            \begin{enumerate}
                \item Лексер разбивает строку на токены, и среди них есть токен открывающей комментирующей скобки $(*$, закрывающей комментирующей скобки $*)$, и токен ошибки, который принимает любой символ
                \item При запуске парсера проводится удаление всех комментариев отдельной функцией removeComments, которая поддерживает количество открытых на данный момент комментирующих скобок. Если количество открытых скобок положительно, то всё, что между ними, опускается, а иначе сохраняется. Функция возвращает Maybe список токенов без многострочных комментариев.

                    Если при удалении комментариев обнаруживаются непарные комментирующие скобки, возвращается Nothing.

                \item Если вернулось Nothing, то выводится сообщение об ошибке

                    Иначе токены без многострочных комментариев передаются дальше, где сначала ищутся оставшиеся ошибки, а затем запускается сам парсер.
            \end{enumerate}
		\item
            Парсер, основанный на yacc.

            В нём также используется лексер Alex, поэтому реализация такая же, как в пункте 1.
		\item
            Парсер, основанный на Парсер-Комбинаторах.

            Имеет встроенную поддержку многострочных комментариев с использованием Text.Parsec.Language
	\end{enumerate}
\end{document}	

